\documentclass[draft]{article}
\usepackage{graphicx}


\title {Physics Project Draft}
\author{Elliott G Charlton - 18213621}

%Defined Commands
\newcommand{\pra}{\medskip\\}


\begin{document}
	\begin{titlepage}
	\centering
	
	{\Huge\textbf{ Low-frequency studies of millisecond-pulsars}} \\
 \bigskip\bigskip
		{\LARGE Elliott G Charlton} \\
		\bigskip
		{\LARGE 18213621}\\
		
\end{titlepage}

\maketitle

\tableofcontents{}
\newpage
\section{Introduction}

\section{Background}
\subsection{The Murchison Wildfield Array (MWA)}

The Murchison Widefield Array (MWA) is an array made up 128 tiles where one tile is made up of sixteen dual polarization dipoles arranged into a 4x4 grid,1.1m apart underneath the dipoles lays a 5x5m mesh screen. Each tile is connected to a Analog beamformer. Beamforming is a signal processing technique which adds delays to each signal received so that they all line up to form one beam for the associated tile. The beamformer produces two signals in X and Y direction  which are sent to the receiver (Tingay, et al. 2013). \pra
Using an extra step in the signal and data processing pipeline to collect voltages from the dipoles, called the Voltage Capture System (VCS) capturing the voltages allows much better (????). The VCS gives an observable bandwidth of 30.72MHz with 24*1.28 MHz sub-bands and has a range of 80-300MHz (Tremblay et al. 2015). \pra
Once data on the VCS has been recorded there are two options with which to process data, called Coherent and Incoherent beamforming. The incoherent beam has a field of view (FoV) of roughly 30 degrees. This is processed by taking all of the signals and summing them, the sensitivity for a single tile is increased by Sqrt(N) where N is the total number of tiles. A Coherent beam has very narrow FoV roughly 2 arc minutes. This is done by adding the signals from each tile before detection and increases sensitivity by a factor of N. Each type of processing has its advantages.\pra
The coherent beam is better for single observations of pulsars due to its increased sensitivity. The incoherent beams larger FoV allows it to be better suited to surveying larger portions of the sky.

\subsection{MWAs Pulsar Science?? }
gvsds
\subsection{Introduction to Pulsars}
\subsection{Millisecond-Pulsars}

\section{Experimental Method?/Analysis?}

\section{Results}

\section{Discussion}
\subsection{Interpretation of Results}
\subsection{Future work}

\section{Conclusion}

\section{References}






\end{document}
